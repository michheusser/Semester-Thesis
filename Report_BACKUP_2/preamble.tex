%!TEX root = Report.tex
% Acknowledgment -----------------------------
\section*{Acknowledgments}

Aknowledgements come here...


 \cleardoublepage

%---------------------------------------------------------------------------
% Table of contents

 \setcounter{tocdepth}{2}
 \tableofcontents

 %\cleardoublepage

%---------------------------------------------------------------------------
% Lists

 \listoffigures % Creates list of all figures used in document
 \cleardoublepage

 % \listoftables  % Creates list of all tables inserted
 %\cleardoublepage

%---------------------------------------------------------------------------
% Abstract

\section*{Abstract}
 \addcontentsline{toc}{section}{Abstract}

System identification is the concept of developing mathematical models of dynamical systems from experimental data. In this project, we consider data from experimental measurements performed on a motorcycle helmet to derive acoustic models for active noise control. High noise levels in helmets are a major health concerns for occupational motorcyclists. Providing good noise attenuation would lower the risk of hearing loss.\\

Recent work on active noise control for motorcycle helmets suggests that combining feedback and predictive feedforward control techniques can lead to promising results. Such methods involve two acoustic models characterising the relationship of the sound at different points of the  helmet. The quality of these models is crucial for the control performance.\\

In this project we will apply identification techniques to derive such models from given data. The data consists of microphone recordings from a series of experiments involving two different types acoustic excitations: white noise and swept sine signals. 

%---------------------------------------------------------------------------
% Symbols

%\chapter*{Symbols}\label{chap:symbole}
% \addcontentsline{toc}{chapter}{Symbols}
%
%\section*{Symbols}
%\begin{tabbing}
% \hspace*{3cm} \= \kill
%  $\phi, \theta, \psi$	\> roll, pitch and yaw angle \\[0.5ex] 					
%  $b$				\> gyroscope bias \\[0.5ex]										
%  $\Omega_m$		\> 3-axis gyroscope measurement \\[0.5ex]   		
% \end{tabbing}
%
%\section*{Indices}
%\begin{tabbing}
% \hspace*{1.6cm}  \= \kill
% $x$ \> x axis \\[0.5ex]
% $y$ \> y axis \\[0.5ex]
% \end{tabbing}
%
%\section*{Acronyms and Abbreviations}
%\begin{tabbing}
% \hspace*{1.6cm}  \= \kill
% ETH \> Eidgenoessische Technische Hochschule \\[0.5ex]
% EKF \> Extended Kalman Filter \\[0.5ex]
% IMU \> Inertial Measurement Unit \\[0.5ex]
% UAV \> Unmanned Aerial Vehicle \\[0.5ex]
% UKF \> Unscented Kalman Filter \\[0.5ex]
%\end{tabbing}
%
% \cleardoublepage

%---------------------------------------------------------------------------

%!TEX root = Report.tex

\chapter*{Preface}

I would like to thank Mr. Tony Wood, MSc. ETH Zurich, Ph.D. Student ETH Zurich, who was a very helpful advisor.

\cleardoublepage

%---------------------------------------------------------------------------




\setcounter{tocdepth}{2}
\tableofcontents

\cleardoublepage


%---------------------------------------------------------------------------
%% Abstract
%
%\chapter*{Zusammenfassung}
% \addcontentsline{toc}{chapter}{Zusammenfassung}
%
%Bla bla \dots
%
\cleardoublepage

\chapter*{Abstract}
\addcontentsline{toc}{chapter}{Abstract}
System identification is the concept of developing mathematical models of dynamical systems from experimental data. In this project, we consider data from experimental measurements performed on a motorcycle helmet to derive acoustic models for active noise control. High noise levels in helmets are a major health concerns for occupational motorcyclists. Providing good noise attenuation would lower the risk of hearing loss.\\

Recent work on active noise control for motorcycle helmets suggests, that combining feedback and predictive feed-forward control techniques can lead to promising results. Such methods involve two acoustic models, which characterize the relationship of the sound at different points in the helmet. The quality of these models is crucial for the control performance. The data used in this report consists of microphone recordings from a series of experiments involving two different types acoustic excitations: white noise and swept sine signals. Two discrete system identification techniques were used to derive such models from given data. The first technique is non-parametric and calculates the discrete data of an empirical transfer function, which is then smoothed to reduce the effects of noise. The second one uses subspace identification techniques, specifically the N4SID algorithm, which derives a parametric state-space description of the models from the previously calculated smoothed empirical functions calculated before. 

\cleardoublepage
